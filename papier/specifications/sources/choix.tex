\subsection{Possibilités d'architecture}
Les possibilités de fonctionnement étudiées sont les suivantes :\\

\vspace{0.5cm}
Le serveur émet en continu les valeurs des capteurs dont il dispose.\\
\textbf{Avantages :}
\begin{itemize}
\item Tous les capteurs disponibles sont émis par le serveur, pas d'erreur avec une demande par le client sur un capteur spécifique ;
\end{itemize}

\textbf{Inconvénients :}
\begin{itemize}
\item Le serveur va envoyer beaucoup d'informations dont certaines qui ne seront pas utilisées, ceci est coûteux en énergie pour un téléphone dont l'autonomie est limitée;
\item Le client peut manquer des informations et devoir attendre que le serveur émette à nouveau les valeurs du capteur. Ceci induirait une perte de temps et réduirait la réactivité du système.
\end{itemize}

\vspace{0.5cm}
Le client demande au serveur d'émettre les données d'un unique capteur, puis ce dernier envoie les données en continu.
\textbf{Avantages :}
\begin{itemize}
\item Le client ne perdra plus de temps à attendre que le bon capteur soit émis ;
\end{itemize}

\textbf{Inconvénients :}
\begin{itemize}
\item Le serveur va potentiellement envoyer des données qui ne seront pas traitées par le client car elles pourraient être émises trop rapidement. On a toujours un gaspillage d'énergie ;
\item Le client peut demander un capteur inexistant sur l'appareil, entrainant un dysfonctionnement du programme.
\end{itemize}

\vspace{0.5cm}
Afin de concilier le meilleur de ces solutions, le fonctionnement est le suivant :\\
\vspace{0.5mm}
Le serveur attend la connexion d'un client, ce-dernier spécifie le capteur dont il souhaite récupérer les données. Pour obtenir ces données, le client envoie une requête à laquelle le serveur répond.
 
\textbf{Avantages :}
\begin{itemize}
\item La consommation d'énergie côté serveur sera gérée efficacement, aucune information ne sera envoyée si le client n'en fait pas la demande explicite ;
\item Le client ne manquera aucune donnée.
\end{itemize}

\textbf{Inconvénients :}
\begin{itemize}
\item Le serveur devra envoyer un message d'erreur si une requête sur un capteur inexistant est faite.
\end{itemize}