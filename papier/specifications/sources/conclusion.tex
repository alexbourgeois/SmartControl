Le système est fonctionnel en l'état. La manière dont il est conçut permet de l'intégrer facilement dans un projet. Toutes les intéractions se font avec une unique classe que ce soit côté client ou côté serveur. L'ajout de nouveau capteur est relativement aisée, il suffit de rajouter dans le code les capteurs dans le switch du traitement des messages. 
Les différents tests que j'ai effectué avec plusieurs smartphones (bas de gamme et haut de gamme) montrent des différences de performances. Certains téléphones ne semblent pas pouvoir accéder à la même vitesse à leur capteur : sur certains bas de gamme il est possible de lire plus rapidement les données que sur d'autres plus haut de gamme. Cela dépend vraiment du téléphone et je n'ai pas réussi à expliquer ces différences. Il pourait être intéressant d'installer des versions différentes d'Android sur un même téléphone pour savoir si ce problème vient de la ROM utlisée ou des capteurs internes.