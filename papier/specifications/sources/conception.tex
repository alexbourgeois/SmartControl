\subsection{Généralités}

\begin{itemize}
\item  La connexion entre le client et le serveur se fait en Bluetooth. \\
\textbf{Pourquoi Bluetooth ?}\\
Etant donné que le serveur se situera sur un appareil android, la notion d'économie d'energie est à prendre en compte car ces appareils fonctionnement sur batterie. Dans ce sens le Bluetooth est une technologie rapide et plus économe en énergie que le Wifi. De plus la liaison sera directe et ne sera pas encombrée comme pourrait l'être une connexion Wifi (de nombreuses trames circulant sur le réseau);

\item Le système est développé entièrement grâce à la bibliothèque Qt.\\
\textbf{Pourquoi Qt ?}\\
Qt est multi-plateforme (windows, linux, OSX, android) ce qui facilite grandement le développement du serveur et du client par l'utilisation d'un langage unique : le C++. De plus Qt utilise le paradigme de la programmation évenementielle qui permet une meilleure lisibilité du code par l'utilisation de signaux et de slots. Le projet étant de développer une API, les classes seront facilement utilisables dans un autre projet en connectant simplement leurs slots et signaux ;\\

\item La société Qt s'oriente vers la 3D avec la création d'un moteur de rendu 3D donc ce projet permettra d'en explorer quelques possibilités comme la manipulation de caméra via l'utilisation des données capteur issues du téléphone, ceci servira comme preuve de fonctionnement de l'application.\\
\end{itemize}

\subsection{Protocole de communication}
Il existe 4 types de requêtes entre le client et le serveur:
\begin{itemize}
\item Établir la connexion;
\item Spécifier le capteur souhaité ;
\item Demander la valeur du capteur;
\item Fermer la connexion.
\end{itemize}

\subsubsection{Établissement de la connexion}
\begin{itemize}
\item Le client fait une requête de connexion;\\
\item Le serveur y répond;\\
\item Le client est connecté ou non selon la réponse du serveur.\\
\end{itemize}

\subsubsection{Spécification du capteur}
\begin{itemize}
\item Le client doit choisir entre les capteurs disponibles via l'IHM. Dans la version actuelle du code il existe 3 capteurs : Rotation (numéro 1), Gyroscope (numéro 2) et Acceléromètre (numéro 3). Le client choisit ensuite l'interval de mise à jour du capteur (pour son timer interne). La requête envoyée est de la forme "x y" avec x le numéro du capteur et y l'interval de temps (ce-dernier est envoyé à titre informatif au serveur, il ne l'utilisera pas); \\
\item Le serveur répond "ready" si le capteur a été correctement créé et un message d'erreur si le capteur n'existe pas.\\
\end{itemize}

\subsubsection{Demande de valeur}
\begin{itemize}
\item Ayant précedemment choisi le capteur qu'il désirait, le client envoie simplement "value" au serveur. La fréquence d'envoie de cette requête est déterminée par la valeur d'interval saisie lors du choix du capteur ;
\item Le serveur va répondre à cette requête par une réponse du type : "c: x y z" avec c : capteur choisi, x : valeur sur l'axe X, y : valeur sur l'axe Y et z : valeur sur l'axe Z.\\
\end{itemize}

\subsubsection{Fermeture de la connexion}
\begin{itemize}
\item Le client fait une requête de fermeture;\\
\item Le serveur y répond et clot la connexion.\\
\end{itemize}