\documentclass[a4paper,12pt]{article}
\usepackage[utf8]{inputenc}
\usepackage[francais]{babel}

\usepackage{titling}

\setlength{\droptitle}{-10em}

\title{\textbf{Proposition de sujet de PAO}}
\author{BOURGEOIS Alex}
\date \today

\begin{document}

\maketitle

\section{Préambule}
Le présent document entre dans le cadre des Projets d'Approfondissement et d'Ouverture (PAO) du semestre 4.2. Il est une proposition de sujet de PAO.

\section{Concept}
L'idée du projet est de développer une API permettant de récupérer, sur un ordinateur, les données des capteurs d'un smartphone sous Android. Le but étant de pouvoir réutiliser ces données afin d'intéragir avec une application \textit{desktop}. Dans ce sens le smartphone sera utilisable comme contrôleur pour diverses applications grâce aux données de l'accéléromètre, du gyroscope, de l'écran tactile, etc.\vspace{0.5cm}\\ 
Utilisations : 
\begin{itemize}
\item Associer les mouvements du téléphone à ceux d'un objet 3D visualisé sur l'application \textit{desktop}.
\item Smartphone utilisé de la même manière que le \textit{Gamepad} de la \textit{Wii U}.
\item De nombreuses autres applications sont possibles.
\end{itemize}

\section{Spécificités}

Caractéristiques techniques du sujet :
\begin{itemize}
\item Utilisation d'un smartphone sous Android et d'un ordinateur sous Linux.
\item Utilisation de Qt pour développer l'application pour smartphone et l'application \textit{desktop}.
\item Emploi de la technologie \textit{Bluetooth} pour effectuer la liaison avec l'ordinateur.
\end{itemize}
\vspace{0.5cm}

Caractéristiques administratives du sujet :
\begin{itemize}
\item Réalisable par 1 ou 2 personnes.
\item Crédits attribués : 1.5 ou 3 à discuter avec l'enseignant.
\end{itemize}

\end{document}